\documentclass[]{article}
\usepackage{lmodern}
\usepackage{amssymb,amsmath}
\usepackage{ifxetex,ifluatex}
\usepackage{fixltx2e} % provides \textsubscript
\ifnum 0\ifxetex 1\fi\ifluatex 1\fi=0 % if pdftex
  \usepackage[T1]{fontenc}
  \usepackage[utf8]{inputenc}
\else % if luatex or xelatex
  \ifxetex
    \usepackage{mathspec}
  \else
    \usepackage{fontspec}
  \fi
  \defaultfontfeatures{Ligatures=TeX,Scale=MatchLowercase}
\fi
% use upquote if available, for straight quotes in verbatim environments
\IfFileExists{upquote.sty}{\usepackage{upquote}}{}
% use microtype if available
\IfFileExists{microtype.sty}{%
\usepackage{microtype}
\UseMicrotypeSet[protrusion]{basicmath} % disable protrusion for tt fonts
}{}
\usepackage[margin=1in]{geometry}
\usepackage{hyperref}
\hypersetup{unicode=true,
            pdftitle={Color schemes in accordance with the ICAE corporate design},
            pdfauthor={Claudius Gräbner},
            pdfborder={0 0 0},
            breaklinks=true}
\urlstyle{same}  % don't use monospace font for urls
\usepackage{color}
\usepackage{fancyvrb}
\newcommand{\VerbBar}{|}
\newcommand{\VERB}{\Verb[commandchars=\\\{\}]}
\DefineVerbatimEnvironment{Highlighting}{Verbatim}{commandchars=\\\{\}}
% Add ',fontsize=\small' for more characters per line
\usepackage{framed}
\definecolor{shadecolor}{RGB}{248,248,248}
\newenvironment{Shaded}{\begin{snugshade}}{\end{snugshade}}
\newcommand{\KeywordTok}[1]{\textcolor[rgb]{0.13,0.29,0.53}{\textbf{#1}}}
\newcommand{\DataTypeTok}[1]{\textcolor[rgb]{0.13,0.29,0.53}{#1}}
\newcommand{\DecValTok}[1]{\textcolor[rgb]{0.00,0.00,0.81}{#1}}
\newcommand{\BaseNTok}[1]{\textcolor[rgb]{0.00,0.00,0.81}{#1}}
\newcommand{\FloatTok}[1]{\textcolor[rgb]{0.00,0.00,0.81}{#1}}
\newcommand{\ConstantTok}[1]{\textcolor[rgb]{0.00,0.00,0.00}{#1}}
\newcommand{\CharTok}[1]{\textcolor[rgb]{0.31,0.60,0.02}{#1}}
\newcommand{\SpecialCharTok}[1]{\textcolor[rgb]{0.00,0.00,0.00}{#1}}
\newcommand{\StringTok}[1]{\textcolor[rgb]{0.31,0.60,0.02}{#1}}
\newcommand{\VerbatimStringTok}[1]{\textcolor[rgb]{0.31,0.60,0.02}{#1}}
\newcommand{\SpecialStringTok}[1]{\textcolor[rgb]{0.31,0.60,0.02}{#1}}
\newcommand{\ImportTok}[1]{#1}
\newcommand{\CommentTok}[1]{\textcolor[rgb]{0.56,0.35,0.01}{\textit{#1}}}
\newcommand{\DocumentationTok}[1]{\textcolor[rgb]{0.56,0.35,0.01}{\textbf{\textit{#1}}}}
\newcommand{\AnnotationTok}[1]{\textcolor[rgb]{0.56,0.35,0.01}{\textbf{\textit{#1}}}}
\newcommand{\CommentVarTok}[1]{\textcolor[rgb]{0.56,0.35,0.01}{\textbf{\textit{#1}}}}
\newcommand{\OtherTok}[1]{\textcolor[rgb]{0.56,0.35,0.01}{#1}}
\newcommand{\FunctionTok}[1]{\textcolor[rgb]{0.00,0.00,0.00}{#1}}
\newcommand{\VariableTok}[1]{\textcolor[rgb]{0.00,0.00,0.00}{#1}}
\newcommand{\ControlFlowTok}[1]{\textcolor[rgb]{0.13,0.29,0.53}{\textbf{#1}}}
\newcommand{\OperatorTok}[1]{\textcolor[rgb]{0.81,0.36,0.00}{\textbf{#1}}}
\newcommand{\BuiltInTok}[1]{#1}
\newcommand{\ExtensionTok}[1]{#1}
\newcommand{\PreprocessorTok}[1]{\textcolor[rgb]{0.56,0.35,0.01}{\textit{#1}}}
\newcommand{\AttributeTok}[1]{\textcolor[rgb]{0.77,0.63,0.00}{#1}}
\newcommand{\RegionMarkerTok}[1]{#1}
\newcommand{\InformationTok}[1]{\textcolor[rgb]{0.56,0.35,0.01}{\textbf{\textit{#1}}}}
\newcommand{\WarningTok}[1]{\textcolor[rgb]{0.56,0.35,0.01}{\textbf{\textit{#1}}}}
\newcommand{\AlertTok}[1]{\textcolor[rgb]{0.94,0.16,0.16}{#1}}
\newcommand{\ErrorTok}[1]{\textcolor[rgb]{0.64,0.00,0.00}{\textbf{#1}}}
\newcommand{\NormalTok}[1]{#1}
\usepackage{graphicx,grffile}
\makeatletter
\def\maxwidth{\ifdim\Gin@nat@width>\linewidth\linewidth\else\Gin@nat@width\fi}
\def\maxheight{\ifdim\Gin@nat@height>\textheight\textheight\else\Gin@nat@height\fi}
\makeatother
% Scale images if necessary, so that they will not overflow the page
% margins by default, and it is still possible to overwrite the defaults
% using explicit options in \includegraphics[width, height, ...]{}
\setkeys{Gin}{width=\maxwidth,height=\maxheight,keepaspectratio}
\IfFileExists{parskip.sty}{%
\usepackage{parskip}
}{% else
\setlength{\parindent}{0pt}
\setlength{\parskip}{6pt plus 2pt minus 1pt}
}
\setlength{\emergencystretch}{3em}  % prevent overfull lines
\providecommand{\tightlist}{%
  \setlength{\itemsep}{0pt}\setlength{\parskip}{0pt}}
\setcounter{secnumdepth}{0}
% Redefines (sub)paragraphs to behave more like sections
\ifx\paragraph\undefined\else
\let\oldparagraph\paragraph
\renewcommand{\paragraph}[1]{\oldparagraph{#1}\mbox{}}
\fi
\ifx\subparagraph\undefined\else
\let\oldsubparagraph\subparagraph
\renewcommand{\subparagraph}[1]{\oldsubparagraph{#1}\mbox{}}
\fi

%%% Use protect on footnotes to avoid problems with footnotes in titles
\let\rmarkdownfootnote\footnote%
\def\footnote{\protect\rmarkdownfootnote}

%%% Change title format to be more compact
\usepackage{titling}

% Create subtitle command for use in maketitle
\newcommand{\subtitle}[1]{
  \posttitle{
    \begin{center}\large#1\end{center}
    }
}

\setlength{\droptitle}{-2em}

  \title{Color schemes in accordance with the ICAE corporate design}
    \pretitle{\vspace{\droptitle}\centering\huge}
  \posttitle{\par}
    \author{Claudius Gräbner}
    \preauthor{\centering\large\emph}
  \postauthor{\par}
      \predate{\centering\large\emph}
  \postdate{\par}
    \date{2019-06-09}


\begin{document}
\maketitle

\section{Introduction}\label{introduction}

This package provides some functions that make it easy to create figures
in accordance with the corporate design colors of the Institute for
Comprehensive Analysis of the Economy (ICAE) at the Johannes Kepler
University in Linz.\footnote{More information about the ICAE can be
  found
  \href{https://www.jku.at/en/institute-for-comprehensive-analysis-of-the-economy/}{here}.}

The package is still under development. Here is just a quick overview
over its functions. Note that all exported function already have a
complete documentation so you might refer to the help function for more
details.

For the following porpuses, this example code and the packages
\texttt{dplyr} as well as \texttt{ggplot2} are used:

\begin{Shaded}
\begin{Highlighting}[]
\NormalTok{x <-}\StringTok{ }\KeywordTok{seq}\NormalTok{(}\DecValTok{1}\NormalTok{,}\DecValTok{10}\NormalTok{, }\DataTypeTok{length.out=}\DecValTok{30}\NormalTok{)}
\NormalTok{y <-}\StringTok{ }\KeywordTok{rep}\NormalTok{(}\DecValTok{1}\NormalTok{, }\KeywordTok{length}\NormalTok{(x))}
\NormalTok{data <-}\StringTok{ }\KeywordTok{data.frame}\NormalTok{(}\DataTypeTok{x=}\NormalTok{x, }\DataTypeTok{y=}\NormalTok{y)}
\NormalTok{palettes <-}\StringTok{ }\KeywordTok{c}\NormalTok{(}\StringTok{"main"}\NormalTok{, }\StringTok{"cool"}\NormalTok{, }\StringTok{"hot"}\NormalTok{, }\StringTok{"mixed"}\NormalTok{)}
\KeywordTok{library}\NormalTok{(dplyr)}
\KeywordTok{library}\NormalTok{(ggplot2)}
\KeywordTok{library}\NormalTok{(icaeDesign)}
\end{Highlighting}
\end{Shaded}

\section{The palettes}\label{the-palettes}

Currently, the package contains the following four palettes:

\includegraphics{icae-vignette_files/figure-latex/unnamed-chunk-2-1.pdf}
\includegraphics{icae-vignette_files/figure-latex/unnamed-chunk-2-2.pdf}
\includegraphics{icae-vignette_files/figure-latex/unnamed-chunk-2-3.pdf}
\includegraphics{icae-vignette_files/figure-latex/unnamed-chunk-2-4.pdf}

All of them come with both continuous and discrete versions.

\subsection{Using the palettes}\label{using-the-palettes}

The key functions are \texttt{scale\_color\_icae()} and
\texttt{scale\_fill\_icae()}, which allow to control the \texttt{color}
and \texttt{fill} aesthetics in ggplot2 objects respectively. Please
note that currently only the application to ggplot2 objects is
supported.

In case of a barplot you might use \texttt{scale\_fill\_icae()} as
follows:

\begin{Shaded}
\begin{Highlighting}[]
\KeywordTok{data}\NormalTok{(}\StringTok{"iris"}\NormalTok{)}
\NormalTok{iris }\OperatorTok
\StringTok{  }\KeywordTok{group_by}\NormalTok{(Species) }\OperatorTok
\StringTok{  }\KeywordTok{summarise_all}\NormalTok{(mean) }\OperatorTok
\StringTok{  }\KeywordTok{ungroup}\NormalTok{() }\OperatorTok
\StringTok{  }\KeywordTok{ggplot}\NormalTok{(.) }\OperatorTok{+}\StringTok{ }
\StringTok{  }\KeywordTok{geom_bar}\NormalTok{(}
    \KeywordTok{aes}\NormalTok{(}\DataTypeTok{x=}\NormalTok{Species, }\DataTypeTok{y=}\NormalTok{Petal.Width, }\DataTypeTok{fill=}\NormalTok{Species), }
    \DataTypeTok{stat =} \StringTok{"identity"}
\NormalTok{    ) }\OperatorTok{+}\StringTok{ }
\StringTok{  }\KeywordTok{scale_fill_icae}\NormalTok{() }\OperatorTok{+}\StringTok{ }
\StringTok{  }\KeywordTok{theme_minimal}\NormalTok{()}
\end{Highlighting}
\end{Shaded}

\includegraphics{icae-vignette_files/figure-latex/unnamed-chunk-3-1.pdf}

Note that by default the functions return a discrete scale and for
continuous cases you need to set \texttt{discrete\ =\ FALSE} explicitly.

\begin{Shaded}
\begin{Highlighting}[]
\NormalTok{iris }\OperatorTok
\StringTok{  }\KeywordTok{ggplot}\NormalTok{(.) }\OperatorTok{+}\StringTok{ }\KeywordTok{geom_point}\NormalTok{(}
    \KeywordTok{aes}\NormalTok{(}\DataTypeTok{x=}\NormalTok{Sepal.Length, }\DataTypeTok{color=}\NormalTok{Sepal.Width, }\DataTypeTok{y=}\NormalTok{Petal.Length)}
\NormalTok{    ) }\OperatorTok{+}
\StringTok{  }\KeywordTok{scale_color_icae}\NormalTok{(}\DataTypeTok{discrete =}\NormalTok{ F) }\OperatorTok{+}\StringTok{ }
\StringTok{  }\KeywordTok{theme_minimal}\NormalTok{()}
\end{Highlighting}
\end{Shaded}

\includegraphics{icae-vignette_files/figure-latex/unnamed-chunk-4-1.pdf}

For distinguishing discrete categories via color, such as countries, I
found the \texttt{mixed} palette superior to \texttt{main}:

\begin{Shaded}
\begin{Highlighting}[]
\NormalTok{iris }\OperatorTok
\StringTok{  }\KeywordTok{ggplot}\NormalTok{(.) }\OperatorTok{+}\StringTok{ }\KeywordTok{geom_point}\NormalTok{(}
    \KeywordTok{aes}\NormalTok{(}\DataTypeTok{x=}\NormalTok{Sepal.Width, }\DataTypeTok{y=}\NormalTok{Sepal.Length, }\DataTypeTok{color=}\NormalTok{Species)}
\NormalTok{    ) }\OperatorTok{+}
\StringTok{  }\KeywordTok{scale_color_icae}\NormalTok{(}\DataTypeTok{palette =} \StringTok{"mixed"}\NormalTok{, }\DataTypeTok{discrete =}\NormalTok{ T) }\OperatorTok{+}\StringTok{ }
\StringTok{  }\KeywordTok{theme_minimal}\NormalTok{()}
\end{Highlighting}
\end{Shaded}

\includegraphics{icae-vignette_files/figure-latex/unnamed-chunk-5-1.pdf}

\section{The new ggplot theme}\label{the-new-ggplot-theme}

The package also features a new theme for \texttt{ggplot2} objects,
called \texttt{theme\_icae}. It includes a number of changes that I
found useful in improving plot appearance in general, but it is not
specifically built to fit the ICAE color scheme.

\begin{Shaded}
\begin{Highlighting}[]
\NormalTok{classic_theme_plot <-}\StringTok{ }\NormalTok{iris }\OperatorTok
\StringTok{  }\KeywordTok{ggplot}\NormalTok{(.) }\OperatorTok{+}\StringTok{ }\KeywordTok{geom_point}\NormalTok{(}
    \KeywordTok{aes}\NormalTok{(}\DataTypeTok{x=}\NormalTok{Sepal.Width, }\DataTypeTok{y=}\NormalTok{Sepal.Length, }\DataTypeTok{color=}\NormalTok{Species)}
\NormalTok{    ) }\OperatorTok{+}
\StringTok{  }\KeywordTok{ggtitle}\NormalTok{(}\StringTok{"The standard ggplot theme"}\NormalTok{) }\OperatorTok{+}
\StringTok{  }\KeywordTok{scale_color_icae}\NormalTok{(}\DataTypeTok{palette =} \StringTok{"mixed"}\NormalTok{, }\DataTypeTok{discrete =}\NormalTok{ T)}

\NormalTok{icae_theme_plot <-}\StringTok{ }\NormalTok{iris }\OperatorTok
\StringTok{  }\KeywordTok{ggplot}\NormalTok{(.) }\OperatorTok{+}\StringTok{ }\KeywordTok{geom_point}\NormalTok{(}
    \KeywordTok{aes}\NormalTok{(}\DataTypeTok{x=}\NormalTok{Sepal.Width, }\DataTypeTok{y=}\NormalTok{Sepal.Length, }\DataTypeTok{color=}\NormalTok{Species)}
\NormalTok{    ) }\OperatorTok{+}
\StringTok{  }\KeywordTok{ggtitle}\NormalTok{(}\StringTok{"The theme_icae() appearance"}\NormalTok{) }\OperatorTok{+}
\StringTok{  }\KeywordTok{scale_color_icae}\NormalTok{(}\DataTypeTok{palette =} \StringTok{"mixed"}\NormalTok{, }\DataTypeTok{discrete =}\NormalTok{ T) }\OperatorTok{+}\StringTok{ }
\StringTok{  }\KeywordTok{theme_icae}\NormalTok{()}

\NormalTok{ggpubr}\OperatorTok{::}\KeywordTok{ggarrange}\NormalTok{(classic_theme_plot, icae_theme_plot, }
                  \DataTypeTok{ncol =} \DecValTok{2}\NormalTok{, }\DataTypeTok{nrow =} \DecValTok{1}\NormalTok{, }\DataTypeTok{common.legend =}\NormalTok{ F)}
\end{Highlighting}
\end{Shaded}

\includegraphics{icae-vignette_files/figure-latex/unnamed-chunk-6-1.pdf}

It also features acceptable defaults for non-standard specifications
such as wrapped plots:

\begin{Shaded}
\begin{Highlighting}[]
\NormalTok{classic_theme_plot <-}\StringTok{ }\NormalTok{iris }\OperatorTok
\StringTok{  }\KeywordTok{ggplot}\NormalTok{(.) }\OperatorTok{+}\StringTok{ }\KeywordTok{geom_point}\NormalTok{(}
    \KeywordTok{aes}\NormalTok{(}\DataTypeTok{x=}\NormalTok{Sepal.Width, }\DataTypeTok{y=}\NormalTok{Sepal.Length)}
\NormalTok{    ) }\OperatorTok{+}
\StringTok{  }\KeywordTok{ggtitle}\NormalTok{(}\StringTok{"The standard ggplot theme"}\NormalTok{) }\OperatorTok{+}
\StringTok{  }\KeywordTok{scale_color_icae}\NormalTok{(}\DataTypeTok{palette =} \StringTok{"mixed"}\NormalTok{, }\DataTypeTok{discrete =}\NormalTok{ T) }\OperatorTok{+}\StringTok{ }
\StringTok{  }\KeywordTok{facet_wrap}\NormalTok{(}\OperatorTok{~}\NormalTok{Species)}

\NormalTok{icae_theme_plot <-}\StringTok{ }\NormalTok{iris }\OperatorTok
\StringTok{  }\KeywordTok{ggplot}\NormalTok{(.) }\OperatorTok{+}\StringTok{ }\KeywordTok{geom_point}\NormalTok{(}
    \KeywordTok{aes}\NormalTok{(}\DataTypeTok{x=}\NormalTok{Sepal.Width, }\DataTypeTok{y=}\NormalTok{Sepal.Length)}
\NormalTok{    ) }\OperatorTok{+}
\StringTok{  }\KeywordTok{ggtitle}\NormalTok{(}\StringTok{"The theme_icae() appearance"}\NormalTok{) }\OperatorTok{+}
\StringTok{  }\KeywordTok{scale_color_icae}\NormalTok{(}\DataTypeTok{palette =} \StringTok{"mixed"}\NormalTok{, }\DataTypeTok{discrete =}\NormalTok{ T) }\OperatorTok{+}\StringTok{ }
\StringTok{  }\KeywordTok{facet_wrap}\NormalTok{(}\OperatorTok{~}\NormalTok{Species) }\OperatorTok{+}
\StringTok{  }\KeywordTok{theme_icae}\NormalTok{()}

\NormalTok{ggpubr}\OperatorTok{::}\KeywordTok{ggarrange}\NormalTok{(classic_theme_plot, icae_theme_plot, }
                  \DataTypeTok{ncol =} \DecValTok{1}\NormalTok{, }\DataTypeTok{nrow =} \DecValTok{2}\NormalTok{, }\DataTypeTok{common.legend =}\NormalTok{ F)}
\end{Highlighting}
\end{Shaded}

\includegraphics{icae-vignette_files/figure-latex/unnamed-chunk-7-1.pdf}


\end{document}
